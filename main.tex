\documentclass[stu,12pt]{apa7}
  \usepackage{times}               % Times New Roman Font Face
  \usepackage[american]{babel}      % Localization
  \usepackage[utf8]{inputenc}      % Input Encoding
  \usepackage{hyperref}            % Hyperlinks
  \usepackage{enumitem}            % Additional Enumeration Environment Settings
  \usepackage{geometry}            % Page Layout
  \usepackage{soul}                % Text Highlighting
  \usepackage{graphicx}            % Images
  \usepackage{csquotes}            % Quoting Environment
  \usepackage{bookmark}            % Required by `csquotes'
  \usepackage{mdframed}            % Colorful Tex-Box Environment
  \usepackage[toc]{appendix}       % Appendix
  \usepackage{fancyhdr}            % Headings and Footers
  \usepackage[%
    style=apa,%
    sortcites=true,%
    sorting=nyt%
  ]{biblatex}
  \usepackage{xcolor}

  % Bibliography Setup
  %% Language Mappings
  \DeclareLanguageMapping{english}{english-apa}
  \DeclareLanguageMapping{american}{american-apa}
  %% Bibliography File Path
  \addbibresource{main.bib}
  %% Categories for Specified Bibliography Items
  %%% Category for sources not referenced in-text
  \DeclareBibliographyCategory{consulted}
  \addtocategory{consulted}{noauthor_business_nodate}
  \addtocategory{consulted}{noauthor_college_nodate}
  \addtocategory{consulted}{scire_breaking_2001}
  \addtocategory{consulted}{cline_science_2013}
  \addtocategory{consulted}{bostrom_rethinking_2011}


  % Hyperlink Setup
  \hypersetup{
    colorlinks = true,
    urlcolor = blue,
    linkcolor = blue,
    citecolor = blue
  }

  % Page and Text Layout
  \geometry{%
    a4paper,%
    top=1in,%
    bottom=1in,%
    left=1in,%
    right=1in%
  }
  \setlength{\headheight}{15pt}

  % Header
  \lhead{COM125CG1-M3D1}

  % Title Page
  \title{%
    M3D1: Would You Call Yourself A Listener or a Talker?
  }
  \shorttitle{Module 3 Discussion 1}
  \author{Ashton Hellwig}
  \authorsaffiliations{Department of Mathematics, Front Range Community College}
  \course{COM125: Interpersonal Communication}
  \professor{Richard Thomas}
  \duedate{November 21, 2020 23:59:59 MDT}
  \date{\today}
  \abstract{%
    \textbf{Overview}\\%
    For reflection: Are you a listener? Or, are you a talker? Can you be both
      simultaneously?\\%

    If you are more of a talker than a listener, what topics are you interesting
      in talking about? Do you feel you sometimes share more about yourself than
      is wise and, in particular, on social media?\\%

    Sometimes, we are not really listening to what is being said resulting in
      our jumping to conclusions. Do you experience this often?\\%

    You should spend approximately 4 hours on this assignment.%
  }

\begin{document}
  % Title Page
  \maketitle

  % Initial Post
  \section{Initial Post}
    \subsection*{Instructions}
      \begin{enumerate}
        \item In the video appearing within the topic titled,
          ``Skills of Effective Listening'' we saw the conflict that resulted
          from Erin Brokovich not listening effectively to Ed. Can you relate
          more to Erin or Ed? Do you think gender comes into play?
        \item Based on your personal experiences, you might have some initial
          ideas of what ineffective and effective listening skills are, but
          what is documented in the research? For this discussion, you will
          start by researching ineffective and effective listening skills and
          the differences in listening between women and men. You will need to
          explore enough scholarly articles to create your lists and response.
        \item Specifically define \textit{listening} then create two lists
          based on your research.
          \begin{enumerate}
            \item Effective Listeners Top 5 List: One list will be where you
              rank the 5 most important skills of effective listeners. The
              skill you feel is most important should be ranked as number 1.
            \item Ineffective Listeners Top 5 List: One list will be where you
              rank the 5 biggest downfalls of ineffective listeners. The issue
              you feel is the biggest problem should be ranked as number 1.
          \end{enumerate}
        \item In addition to posting your lists, write a 1--2 paragraph (a
          minimum of three complete sentences per paragraph) summary of your
          research on the topic of how women and men listen differently. Does
          your research align closely or not closely with how you communicate?
          Why or why not?
      \end{enumerate}


    \newpage
    \subsection{Analysis of ``The Bonus Check''}
      In the movie clip of ``The Bonus Check'' we were to view, we see Erin
        \textit{seem to} overreact when she was being told that her bonus check
        would not be the figure that she and Ed had discussed previously. The
        discussion prompts us to discuss how \textit{Erin} was ineffective in
        her listening to her boss, Ed, but I do not see that as the full case.
        In this scenario, Ed's communication is \textit{also off} due to his
        tone making the situation seem more dire than it was
        \parencite{soderbergh_erin_2000}. This was perhaps intentional, hence
        why Ed did not penalize Erin for her ``outburst'' as he began speaking
        by utilizing a somber tone. I, too, would have reacted in the way that
        Erin did (and then, feel completely embarassed afterwards). Not being
        appreciated for the work that you do is something that hurts, as it rips
        away purpose from the menial tasks we do day-to-day.


    \subsection{Listening: What Does it Mean?}
      Personally, I would define listening as ``providing one's full, undivided
        attention, to another individual speaking'' (or playing music).
        Listening is one of, if not \textit{the} most important aspect when it
        comes in interpersonal communication and maintaining strong
        relationships \parencite{bodie_listening_2012}. There are many sources
        we can pull the definition of \textit{listening} from, with almost all
        of them mentioning the importance and postive connoations that listening
        has behind it, while also citing that \textit{understanding} is far more
        important than responding \parencite[pp. 116]{bodie_listening_2012}.

      \subsubsection{The Styles of Listening}
        There are many \textit{styles} of listening, and many of us may identify
          with more than one of them depending on the other individual we are
          speaking with or the present situation.

        \paragraph{People-oriented listeners}
          People-oriented listeners prioritize addressing another person's
            feelings over any other task that may be at hand
            \parencite[pp. 338]{noauthor_communication_2013}.

        \paragraph{Action-oriented listeners}
          Action-oriented listeners are those individuals whom listen best when
            information being given to them by the speaker is well-organized
            and lacks any sort of ``fluff'' not needed to convey the information
            \parencite[pp. 338--339]{noauthor_communication_2013}.

        \paragraph{Content-oriented listeners}
          Content-oriented listeners are just that --- focused on the subbstance
            of the other individual's speaking
            \parencite[pp. 339]{noauthor_communication_2013}.

        \paragraph{Time-oriented listeners}
          A time-oriented listener is more concerned with timelines and
            deadlines than they are with the true \textit{substance} of the
            message being conveyed to them
            \parencite[pp. 339--340]{noauthor_communication_2013}.


      \subsubsection{Effective Listening Skills}
        The listening skills taken below were mainly sourced from Saylor
          Academy's \textit{Communication in the Real World}
          \parencite[pp. 330--345]{noauthor_communication_2013}.

          \begin{description}
            \item[Empathy]
              \parencite[pp. ]{noauthor_communication_2013}.
            \item[Active Listening]
              \parencite[pp. ]{noauthor_communication_2013}.
            \item[Critial-Analysis]
              \parencite[pp. ]{noauthor_communication_2013}.
            \item[Four]
              \parencite[pp. ]{noauthor_communication_2013}.
            \item[Five]
              \parencite[pp. ]{noauthor_communication_2013}.
          \end{description}

      \subsubsection{Ineffective Listening Skills}
          \begin{description}
            \item[Interruption]
              \parencite[pp. 351--352]{noauthor_communication_2013}.
            \item[Distorted Listening]
              \parencite[pp. 352]{noauthor_communication_2013}.
            \item[Eavesdropping]
              \parencite[pp. 353--354]{noauthor_communication_2013}.
            \item[Aggressive Listening]
              \parencite[pp. 354--355]{noauthor_communication_2013}.
            \item[Narcissistic Listening]
              \parencite[pp. 355--356]{noauthor_communication_2013}.
            \item[Pseudo Listening]
              \parencite[pp. 356--357]{noauthor_communication_2013}
          \end{description}


    \subsection{The Differences of How Men and Women Listen}
      As an individual identifying as Gender Fluid I am not entirely sure how
        to speak to the differences between how men and women listen. I, like
        all of us, have had thousands of conversations with many different
        individuals identifying across the Gender Spectrum and every person
        listens in a different way, and I am not entirely convinced that it is
        entirely inherent on one's assigned Gender.

      Perhaps the gender with which one was assigned at birth \textit{could}
        influence prior experiences which would then be the thing that separates
        how one person listens differently than another. Many factors affect
        how we listen aside from gender, including personal traits as well as
        the surrounding atmosphere \parencite[pp. 6]{caspersz_can_nodate}.

    \subsection{A Note on Self-Disclosure}
      It is important to note that self-disclosure is \textit{never} on-sided.
        Once one person discloses information about themselves, the others
        around will react to it (agreeing with or disagreeing with what the
        person showed everyone about themselves)
        \parencite[pp. 463]{noauthor_communication_2013}


  % Replies
  %! TEX root=../main.tex

\section{Responses}
  \subsection{Response 1}
    \begin{quotation}
      Listening is something that comes easier to me than talking.
        As I indicated by my goals, one of which is to be less concise, I do not
        often just volunteer information about myself.  I prefer to listen to
        what others have been thinking about or learn more about them. I can
        relate to Erin because I would probably be wondering why I did not get
        the bonus that they had talked about, although instead of making a fuss
        about it I would probably have just checked out of the situation.
        Overall, the way Ed started the conversation, despite the fact he was
        trying to surprise her, would have caused anyone to be a bit upset.
        I do not really think that gender had anything to do with this situation
        and the reactions they had. It seems to me that this would have been
        more influenced by personality, not gender.

      I believe that we can define listening as processing information or
        emotions from others.  One article I read when studying effective and
        not effective listening skills, which talks about things effective
        listeners do states, “These behaviors include responses such as
        ``uh huh'' and ``hmmm,'' as well as other nonverbal behaviors including
        nodding, smiling, and adjusting one's posture. Other active listening
        behaviors include asking questions, making eye contact, and not
        interrupting the speaker” (Fedesco, 2015).  Asking questions can help
        you to not only better remember information, but also better understand
        it. That article goes on to say that the more you use those behaviors
        indicates how invested you are in a certain conversation. Another
        article I read makes this claim that being a good listening can offer,
        ``A greater number of friends and social networks, improved self-esteem
        and confidence, higher grades at school and in academic work, and even
        better health and general well-being'' (Listening Skills). This means
        that being a good listener can provide a large number of benefits. Being
        a good or bad listener would definitely affect your grades in school.

      As I tried to find a difference between male and female in terms of
        listening, I did not find anything that indicated one listens better
        than the other.  I read an article which states, “Despite activating
        different activity centers within the brain, genders perform equally on
        measures of cognitive function. This means that although we listen and
        assimilate information differently, the difference does not appear to
        affect cognition or our ability to listen” (McCormick,2018). Therefore,
        our brains may function in a different way, but that does not indicate
        that either men or women listen better simply based on their gender.
        I would say my research aligns pretty well with how I communicate, with
        exception to the occasionally accidental interruption.  Therefore, I
        would still classify myself as a good listener.  I often ask questions,
        make eye contact, and am sympathetic to the person speaking.

      Effective Listeners Top 5 List:
      \begin{enumerate}
        \item Asking questions
        \item Making eye contact
        \item Not interrupting
        \item Smiling
        \item Being polite
      \end{enumerate}

      Ineffective Listeners Top 5 List
      \begin{enumerate}
        \item Interrupting
        \item Looking away
        \item Crossing their arms
        \item Not reacting with smiles or frowns (not sympathetic)
        \item Often changing the subject
      \end{enumerate}
    \end{quotation}

    \paragraph{This is a response to Thora Smith on Post ID 43465667}
      I completely agree with you on the side of how most \textit{anyone} would
        be upset if being approached the same way Erin was prior to the
        ``surprise/joke'' about her bonus amount was realized. If my boss were
        to have come to me in a similar manner, I am sure I would have reacted
        in the exact same way as Erin did when she felt she was being valued
        by her company less than she was worth.

      I also believe, like you do, that \textit{asking questions} is an
        important part of being an effective listener in \textbf{any} scenario.
        I do have on question about this though (or rather, more a solicitation
        for advice) which maybe you can help me with. When a question regarding
        what someone is telling you comes into your head, do you ask it right
        away (thereby interupting the speaker's conversational flow) or do you
        hold the question until after they are done speaking, when it may
        \textit{no longer be relevant}? This is something I always have
        difficulty with. Sometimes my question is answered by the speaker by
        listening to the rest of the story, but sometimes it is not and when
        I bring it up again at the end of a conversation sometimes the other
        party does not even remember that far back into the conversation to
        answer! Incredibly frustrating on occassion, depending on the situation
        this occurs. What is your opinion on that matter?

  % %! TEX root=../main.tex

\subsection{Response 2}
  \begin{quotation}
    Placeholder.
  \end{quotation}

  \paragraph{This is a response to FIRST LAST on Post ID 00000000}
    Placeholder.



  % Bibliography
  %% Works Cited
  \newpage
  \printbibliography[%
    title={References},%
    heading={bibintoc},%
    notcategory={consulted}%
  ]

  %% Works Consulted
  \newpage
  \nocite{*}
  \printbibliography[%
    title={Additional References},%
    heading={bibintoc},%
    category={consulted}%
  ]
\end{document}
